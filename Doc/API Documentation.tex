\documentclass{article}
\usepackage{hyperref}
\usepackage{listings}
\usepackage{geometry}


\title{API Documentation}
\author{CrocoMarine ROV}

\begin{document}


\maketitle

\tableofcontents
\hypersetup{
    colorlinks=true,
    linkcolor=blue,
    filecolor=magenta,      
    urlcolor=cyan,
}


\section{Introduction}
This API is the program of the MATE NOAA Competition by the CrocoMarine ROV team. It is designed to track objects in video frames and is open-sourced. and NOAA given permission to use it on additional videos.
\section{ObjectTracker API}
============================================

List all the API endpoints along with their descriptions, request methods, and parameters.
\subsection{Overview}
-----------------------

\begin{itemize}
    \item The \textbf{ObjectTracker} class is a Python module designed to track objects in video frames.
    \item It provides two main modes of operation: object detection and tracking.
    \item The class uses YOLOV8 (You Only Look Once) for object detection and tracking.
\end{itemize}

\subsection{Main Functionality}
-----------------------

\begin{itemize}
    \item \textbf{Object Detection Mode (mode=1)}: In this mode, the class uses YOLO to detect objects in each frame and returns the bounding boxes for each.
    \item \textbf{Object Tracking Mode  (mode=0)}: In this mode, the class uses the tracking data from the previous frame to track objects in the current frame.
    \item The tracked objects are then updated with new bounding boxes and track IDs.
    \item The class also provides options to save the tracking data to an Excel file using \textit{\textbf{ExcelHandler}}and video output using the \textit{\textbf{VideoSaver}} class.
\end{itemize}

\subsection{Class Methods}
-----------------------

\begin{itemize}
    \item\textbf{\_init\_ :} 
    
    \textbf{Description :} Initializes the \textbf{ObjectTracker} object.

    \textbf{Parameters :}    
    \begin{itemize}
        \item \textit{model\_path (str):} The path to the YOLOv8 model file.
        \item \textit{confidence\_threshold (float):} The confidence threshold for detection (default: 0.25).
        \item \textit{classes (list):} The list of classes to detect (default: [0]).
        \item \textit{save\_output (bool):} A flag to save output (default: False).
        \item \textit{save\_data (bool):} A flag to save data (default: False).
        \item \textit{origional\_size (bool):} A flag to use original size (default: False).
        \item \textit{mode (int):} The mode of operation (default: 0).
        \item \textit{show\_output (bool):} A flag to show output (default: False).
    \end{itemize}
    
    \item\textbf{run :}
    
    \textbf{Description :} Runs the object tracking process on the specified input video or image.

    \textbf{Parameters :}
    \begin{itemize}
        \item \textit{input\_path (str):} The path to the input video or image.
    \end{itemize}

    \textbf{Returns :} None
    
\end{itemize}

\subsection{Example Usage}
-----------------------

\begin{lstlisting}[language=Python]
    from CrocoMarine_program_2024 import ObjectTracker

    # Create an ObjectTracker object
    detector = ObjectTracker(
        model_path="CrocoMarine_model_2023.pt",
        confidence_threshold=0.4,
        classes=[0],
        save_output=True,
        save_data=True,
        origional_size=False,
        mode=1, # detection mode
        show_output=False
    )
    
    # Run the object tracking process
    detector.run("seafloor_footage.mp4")
\end{lstlisting}

\subsection{Command-Line Interface}
-----------------------

\subsubsection{Overview :}

    The `ObjectTracker` class can also be used through a command-line interface using the following command:

    This will run the API with the specified model file, input video, confidence threshold, classes to detect, and flags to save output and data.

\subsubsection{Example :}
\begin{lstlisting}[language=bash, breaklines=true]
    python main.py --model_path "CrocoMarine_model_2023.pt" -ip "seafloor_footage.mp4" --mode 1  --conf 0.4 -show_output --save_output --save_data --origional_size
\end{lstlisting}

\subsubsection{Arguments  :}

The API uses the `argparse` library to parse command line arguments. The arguments are defined in the `main.py` file, and they are used to configure the `ObjectTracker` class.

\begin{itemize}
    \item {\textbf{--model\_path (required):}} The path to the model file.
    \item {\textbf{--conf (optional, default=0.25):}} The confidence threshold for detection.
    \item {\textbf{--classes (optional, default=[0]):}} The list of classes to detect.
    \item {\textbf{--save\_output (optional, default=False):}} Flag to save output.
    \item {\textbf{--save\_data (optional, default=False):}} Flag to save data.
    \item {\textbf{--origional\_size (optional, default=False):}} Flag to use original size.
    \item {\textbf{--show\_output (optional, default=False):}} Flag to show the model prediction output.
    \item {\textbf{--mode (optional, default=0):}} The mode of operation (choices: 0, 1).
    \item {\textbf{--input\_path (required):}} The path to the input video or image.
\end{itemize}

\subsection{Output Formats}
-----------------------

    The `ObjectTracker` class generates output in the following formats:
    \begin{itemize}
        \item \textbf{Video(.MP4)} : A video file showing the tracked objects.
        \item \textbf{Data(.xlsx)} : A excel file containing information about the tracked objects.
    \end{itemize}


\section{ExcelHandler API Documentation}
============================================

\subsection{Overview :}
-----------------------

    The `ExcelHandler` class is designed to handle Excel file operations, specifically adding data to an internal data list and saving it to an Excel file. This class provides a simple and efficient way to manage Excel data.

\subsection{Main Functionality :}
-----------------------

    The `ExcelHandler` class is designed to handle the following main functionality:

\begin{itemize}
    \item \textbf{Data Addition}
        \begin{itemize}
            \item Description: The `add\_data` method allows you to add rows of data to the internal data list. This data can be in the form of lists, where each list represents a row in the Excel file.
        \end{itemize}
    \item \textbf{Data Saving}
        \begin{itemize}
            \item Description: The `save` method saves the internal data to the specified Excel file. The data is saved in a structured format, with each row representing a list of values.
        \end{itemize}
\end{itemize}
\subsection{Class Methods :}
-----------------------

    The `ExcelHandler` class offers the following methods:

\begin{itemize}
    \item \textbf{\_init\_}
        \begin{itemize}
            \item Description: Initializes the `ExcelHandler` object with a file path.
            \item Parameters:
                \begin{itemize}
                    \item {`file\_path` (str):} The path to the Excel file.
                \end{itemize}
            \item Returns: None
        \end{itemize}
    \item \textbf{add\_data}
        \begin{itemize}
            \item Description: Adds a row of data to the internal data list.
            \item Parameters:
                \begin{itemize}
                    \item `row` (list): A list of values to be added as a row in the Excel file.
                \end{itemize}
            \item Returns: None
        \end{itemize}
    \item \textbf{save}
        \begin{itemize}
            \item Description: Saves the internal data to the Excel file.
            \item Parameters: None
            \item Returns: None
        \end{itemize}
\end{itemize}


\section{VideoSaver API Documentation}
============================================

\subsection{Overview}
-----------------------

The `VideoSaver` class is designed to save video frames to a file. It provides a simple and efficient way to record video from various sources, such as cameras or video processing pipelines.

\subsection{Main Functionality}
-----------------------

\begin{itemize}
    \item The `VideoSaver` class provides a simple and efficient way to record video from various sources, such as cameras or video processing pipelines.
\end{itemize}

\subsection{Class Methods}
-----------------------

\begin{itemize}
    \item \textbf{\_init\_}
        \begin{itemize}
            \item Description: Initializes the `VideoSaver` object with the specified filename, frames per second, and FourCC code.
            \item Parameters:
                \begin{itemize}
                    \item {`filename` (str):} The filename to save the video to.
                    \item {`fps` (float):} The frames per second to save the video at. Default is 15.0.
                    \item {`fourcc` (str):} The FourCC code to use for the video codec. Default is 'mp4v'.
                \end{itemize}
            \item Returns: None
        \end{itemize}
    \end{itemize}


\begin{itemize}
    \item \textbf{start}
        \begin{itemize}
            \item Description: Starts the video saver, creating a `VideoWriter` object with the specified width and height.
            \item Parameters:
                \begin{itemize}
                    \item `width` (int): The width of the video frames.
                    \item `height` (int): The height of the video frames.
                \end{itemize}
            \item Returns: None
        \end{itemize}
    \item \textbf{write}
        \begin{itemize}
            \item Description: Writes a single frame to the video file. If the video saver has not been started, this method does nothing.
            \item Parameters:
                \begin{itemize}
                    \item `frame` (numpy.ndarray): The frame to write.
                \end{itemize}
            \item Returns: None
        \end{itemize}
    \item \textbf{stop}
        \begin{itemize}
            \item Description: Stops the video saver, releasing any system resources associated with the `VideoWriter` object.
            \item Parameters: None
            \item Returns: None
        \end{itemize}
\end{itemize}




\end{document}
